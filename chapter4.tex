\chapter{فرصت‌های تحقیقاتی}
تلاش‌های قبلی عمدتا بر روی مفاهیم کلی پیش بینی و طبقه بندی متمرکز است و تعداد کمی از آنها می‌توانند برای کشف سایر کاربردهای ممکن از این محدوده خارج شوند. با این حال، با آخرین موفقیت در یادگیری ماشین و زیرساخت‌های آن، ممکن است خواسته‌های بالقوه جدیدی در زمینه‌های شبکه ظاهر شود. برخی از فرصت‌ها به شرح زیر معرفی می شوند\cite{ wang2017machine}:

\section{مجموعه‌داده هاي در دسترس}

در دامنه شبکه، به دلیل دسترسی محدود و هزینه آزمایش بالای سیستم‌های شبکه در مقیاس بزرگ، شبیه سازهایی با قابلیت اطمینان کافی، مقیاس پذیری و سرعت بالا مورد نیاز هستند. این موارد به توسعه بیشتر دامنه شبکه کمک می‌کنند و منابع عمومی نیز امکان تحقیق برای عموم را فراهم می‌کنند.

\section{پروتکل شبکه خودکار و طراحی معماری}
با درک عمیق تری از شبکه، محققان به تدریج دریافتند که شبکه موجود محدودیت‌های زیادی دارد. سیستم شبکه کاملا توسط بشر ایجاد شده است. اجزای فعلی شبکه احتمالاً براساس درک افراد در یک لحظه و نه یک نمونه مهندسی اضافه می‌شوند. هنوز فضای کافی برای بهبود عملکرد و کارایی شبکه با طراحی مجدد پروتکل و معماری شبکه وجود دارد. امروزه طراحی پروتکل یا معماری به طور خودکار کاملاً دشوار است. نتایج تولید شده هنوز از امکان طراحی پروتکل دور هستند. پتانسیل ایجاد مولفه‌های جدید شبکه بدون دخالت انسان وجود دارد، که ممکن است درک انسان از سیستم‌های شبکه را تغییر دهد. 


\section{خلاصه}

در این فصل دو فرصت جهت تحقیقات بیشتر معرفی کردیم. مجموعه داده‌های در دسترس دیگران که برای آزمون و شبیه‌سازی بسیار مفید هستند و کار را برای تحقیقات بسیار راحت خواهند کرد. و طراحی پروتکلی خودکار که زیر ساخت شبکه را تغییر خواهد داد. که پتانسیل بسیار بالایی هم دارد.


\newpage