\chapter{سیر تحول و کاربرد یادگیری ماشین}

\section{مهندسی ترافیک}
\subsection{پیش بینی ترافیک}

به عنوان یک مسئله مهم تحقیقاتی، تخمین دقیق میزان ترافیک برای کنترل ازدحام، تخصیص منابع، مسیریابی شبکه و حتی برنامه های پخش زنده مفید است. عمدتا دو جهت تحقیق بسته به مشاهدات مستقیم و غیر مستفیم وجود دارد:
\begin{itemize}
    \item پیش بینی سری‌های زمانی\LTRfootnote{\lr{Time Series Forecasting(TSF)}}
    \item توموگرافی شبکه\LTRfootnote{\lr{Network Tomography}}
    
\end{itemize}
با این حال، اندازه‌گیری مستقیم میزان ترافیک، به ویژه در یک فضای شبکه با سرعت بالا در مقیاس بزرگ، هزینه‌ی زیادی دارد.


تمرکز بسیاری از مطالعات موجود در کاهش هزینه اندازه‌گیری با استفاده از معیارهای غیرمستقیم و نه فقط استفاده از الگوریتم‌های مختلف یادگیری ماشین است. برای حل این مشکل دو روش وجود دارد.  با جستجوی دانش خاص دامنه و الگوهای داده کشف نشده، تلاش بیشتری برای توسعه الگوریتم‌های پیچیده انجام دهیم. روش دیگر از رویکرد یادگیری عمیق انتها به انتها الهام گرفته است. برخی از اطلاعات به دست آمده را به عنوان ورودی مستقیم درنظر گرفته و ویژگی‌ها را با کمک مدل یادگیری به طور خودکار استخراج می‌کند \cite{wang2017machine}.

\subsection{طبقه‌بندی ترافیک}

طبقه بندی ترافیک به عنوان یک مولفه اساسی عملکرد در مدیریت شبکه و سیستم‌های امنیتی، برنامه‌ها و پروتکل‌های شبکه را با جریان ترافیک نگاشت می‌کند. 

دو روش‌ سنتی طبقه بندی ترافیک، رویکرد مبتنی بر پورت\LTRfootnote{\lr{Port}} و رویکرد مبتنی بر پیلود\LTRfootnote{\lr{PayLoad}} است. رویکرد مبتنی بر پورت به دلیل استفاده مجدد و یا غیر صحیح از پورت، ناکارامد است، از طرفی رویکرد مبتنی بر پیلود مشکلات حریم خصوصی ناشی از بازرسی بسته‌ها را دارد، که حتی می‌تواند در حضور ترافیک رمزگذاری شده ناکارامد باشد.

\newpage

در نتیجه، رویکردهای یادگیری ماشین بر اساس ویژگی‌های آماری در سال‌های اخیر به ویژه در حوزه امنیت شبکه به طور گسترده مورد بررسی قرار گرفته است. با این حال، تقریبا نمی‌توان که یادگیری ماشین را به عنوان یک راه حل کامل در نظر بگیریم و آن را در یک محیط عملیاتی دنیای واقعی بکار ببریم. زیرا با طبقه بندی غلط در زمینه امنیت شبکه هزینه زیادی ایجاد می کند. به طور کلی، این مطالعات از سناریوهای شناخته‌شده طبقه بندی شروع می‌شود و تا شرایط دنیای واقعی با ترافیک ناشناخته (به عنوان مثال، ترافیک روز صفر\LTRfootnote{\lr{Zeroday}}) ادامه می‌یابد. این فرایند مطالعه بسیار شبیه به فرایند یادگیری ماشین است که از یادگیری تحت نظارت به یادگیری بدون نظارت و نیمه نظارت تکامل می‌یابد، که می‌تواند به عنوان یک الگوی پیشگام برای وارد کردن یادگیری ماشین در زمینه‌های شبکه تلقی شود.

%\subsection{مسیریابی ترافیک}

\section{کارایی و بهینه‌سازی}
\subsection{کنترل ازدحام}

كنترل ازدحام\LTRfootnote{\lr{Congestion Control}} در شبكه مسئول جریان تعداد بسته‌هاي ورودي به شبكه است. این پایداری شبکه، تعادل در استفاده از منابع و نسبت قابل قبول از دست رفتن بسته را تضمین می‌کند. معماری‌های مختلف شبکه مجموعه مکانیسم‌های کنترل ازدحام خود را به کار می‌گیرند. مشهورترین مکانیزم‌های کنترل ازدحام در پروتکل کنترل انتقال\LTRfootnote{\lr{Transmission Control Protocol(TCP)}} پیاده سازی شده‌اند، زیرا این پروتکل همراه با پروتکل اینترنت\LTRfootnote{\lr{Internet Protocol}} اساس اینترنت فعلی را تشکیل می‌دهند. مکانیزم‌های کنترل ازدحام این پروتکل در سیستم‌های نهایی شبکه کار می‌کنند تا هنگام شناسایی ازدحام، سرعت ارسال بسته را محدود کنند. یکی دیگر از مکانیزم‌های معروف کنترل ازدحام، مدیریت صف است که در داخل گره‌های میانی شبکه (به عنوان مثال سوئیچ‌ها و روترها) برای تکمیل کارکرد این پروتکل عمل می‌کند. پیشرفت‌های کنترل ازدحام در اینترنت و معماری‌های تکاملی شبکه وجود داشته است. با وجود این تلاش‌ها، کاستی‌های مختلفی در زمینه‌هایی مانند طبقه بندی بسته‌های از دست رفته، مدیریت صف و به روزرسانی پنجره ازدحام \LTRfootnote{\lr{CWND}}وجود دارد\cite{boutaba2018comprehensive}.
\newpage




\subsection{مدیریت منابع}
مدیریت منابع شامل کنترل منابع حیاتی شبکه، از جمله پردازنده\LTRfootnote{\lr{CPU}}، حافظه، دیسک، سوئیچ‌ها، روترها، پهنای باند، کانال‌های رادیویی و فرکانس‌های آن است. اینها به طور جمعی یا مستقل برای ارائه خدمات استفاده می‌شوند. ارائه دهندگان خدمات شبکه می‌توانند مقدار مشخصی از منابع را فراهم کنند که تقاضای مورد انتظار برای یک سرویس را تأمین کند. با این حال، پیش بینی تقاضا غیرمعمول نیست، در حالی که تخمین زیاد و بیش از حد منجر زیان شود. بنابراین، یک چالش اساسی در مدیریت منابع، پیش بینی تقاضا و تهیه مجدد منابع به صورت پویا است، به طوری که شبکه در برابر تغییرات تقاضای خدمات مقاوم باشد. با وجود کاربرد گسترده یادگیری ماشین برای پیش بینی و مدیریت منابع در مراکز داده ابری، هنوز هم چالش‌های مختلفی برای شبکه‌های مختلف وجود دارد. در این بررسی، دو نوع چالش را در نظر می‌گیریم\cite{ boutaba2018comprehensive}:
\begin{itemize}
\item کنترل پذیرش\LTRfootnote{\lr{ Admission contro}} یک رویکرد غیر مستقیم برای مدیریت منابع است که نیازی به پیش بینی تقاضا ندارد. هدف در کنترل پذیرش، بهینه سازی استفاده از منابع با نظارت و مدیریت منابع در شبکه است. پذیرش یک درخواست جدید برای ارائه دهنده خدمات شبکه درآمدزایی دارد. با این حال، ممکن است کیفیت خدمات موجود را به دلیل کمبود منابع پایین آید و درآمد خود را از دست بدهد. بنابراین، بین پذیرفتن درخواست های جدید و حفظ کیفیت، یک مصالحه وجود دارد. کنترل پذیرش این چالش را برطرف می‌کند و هدف آن به حداکثر رساندن تعداد درخواست های پذیرفته شده با حفظ کیفیت است.
\item تخصیص منابع\LTRfootnote{\lr{ Resource Allocation}} یک مسئله تصمیم‌گیری است که به طور فعال منابع را مدیریت می‌کند تا یک هدف بلند مدت مانند استفاده از منابع یا درآمد را به حداکثر برساند. چالش اساسی در تخصیص منابع، انطباق منابع برای منافع بلند مدت که غیرقابل پیش بینی هستند. تخصیص منابع نمونه‌ای برای برجسته کردن مزایای یادگیری ماشین است، که می‌تواند تهیه منابع را به روش‌های مختلف یاد گرفته و مدیریت کند.
\end{itemize}

\newpage



\section{امنیت شبکه}

معمول ترین رویکرد امنیتی نظارت بر شبکه برای الگوهای تهدید‌های\LTRfootnote{\lr{Threats}} شناخته شده است. با این حال، شبکه در برابر حملات روز صفر آسیب‌پذیر\LTRfootnote{\lr{Vulnerable}} است. به همین دلیل به اقدامات امنیتی نیاز است و نقش یادگیری ماشین در این راستا به طور گسترده بررسی شده است. 


تلاش‌های موجود، متمرکز بر استفاده از یادگیری ماشین در شناسایی سو استفاده‌ها\LTRfootnote{\lr{Misuse detection}} است که به منظور یادگیری الگوی حمله‌های پیچیده از داده‌های تاریخی و تولید قوانین عمومی است که امکان شناسایی انواع حملات شناخته شده را فراهم می‌کند. همچنین تشخیص ناهنجاری\LTRfootnote{\lr{Anomaly detection}} با استفاده از یادگیری ماشین برای کشف حملات روز صفر بررسی شده است. که شامل یادگیری الگوهای رفتار طبیعی و تشخیص انحراف از وضعیت معمول است\cite{ayoubi2018machine}.


در ابتدا تعجب‌آور است که با وجود تلاش‌های گسترده در تحقیقات دانشگاهی در زمینه تشخیص ناهنجاری، موفقیت چنین سیستم‌هایی در محیط‌های عملیاتی بسیار محدود بوده است. ثابت شده‌است که همان ابزارهای یادگیری ماشین که اساس سیستم‌های تشخیص ناهنجاری است، با موفقیت زیادی کار می‌کنند و بازرسی دائمی داده‌ها در جایی که غیرقابل اجرا است، مورد استفاده قرار می‌گیرند. این موفقیت و شکست به این دلیل بوجود می‌آید که دامنه تشخیص نفوذ، ویژگی‌های خاصی را نشان می‌دهد که استقرار موثر رویکردهای یادگیری ماشین را اساساً دشوارتر از بسیاری از زمینه‌های دیگر می‌کند\cite{sommer2010outside}.


\section{خلاصه}

در این فصل سیر تحول و کاربرد‌های یادگیری ماشین در زمینه‌های مختلف شبکه را دیدیم. ابتدا در زمینه‌ی مهندسی ترافیک به بحث پیش بینی و طبقه‌بندی ترافیک پرداختیم. و از مشکلات و معضلات روش‌های قدیم و جدید صحبت کردیم. سپس به سراغ زمینه کارایی در شبکه ها رفتیم و درمورد کنترل ازدحام بسته‌ها و مدیریت منابع سخت افزاری بحث کردیم. و درنهایت فصل را با دو کاربرد یادگیری ماشین در زمینه امنیت شبکه خاتمه دادیم.





\newpage