\chapter{مقدمه}
\section{مقدمه}

یادگیری ماشین\footnote{\lr{Machine Learning}} یک سیستم را قادر می‌سازد تا داده‌ها را بررسی کند و دانش را استنباط کند. الگوهای آموخته‌شده برای تجزیه و تحلیل داده‌های ناشناخته استفاده می‌شوند، به گونه‌ای که می‌توان آنها را گروه بندی کرد یا به گروه‌های شناخته شده آنها را تقسیم بندی کرد. تکنیک‌های اولیه یادگیری ماشین غیر قابل انعطاف بوده و قادر به تغییری در داده‌های آموزشی نیستند. پیشرفت‌های اخیر در یادگیری ماشین باعث شده این تکنیک‌ها در دنیای واقعی انعطاف پذیر و مقاوم باشند.
\\
موفقیت تکنیک‌های یادگیری ماشین تا حد زیادی به داده‌ها وابسته‌است. داده‌های عظیم در شبکه‌های امروزی وجود دارد که با شبکه‌های در حال ظهوری مانند اینترنت اشیا و میلیاردها دستگاه متصل به آن، رشد بیشتری خواهدکرد. به همین دلیل کاربرد یادگیری ماشین بیشتر خواهدشد که نه تنها الگوهای پنهان را شناسایی کند، بلکه می‌تواند برای یادگیری و درک فرایندهای تولید داده نیز به کار رود.
\\
علیرغم پیشرفت‌ها در یادگیری ماشین، عملیات و مدیریت شبکه همچنان دشوار است و خطاهای شبکه عمدتا به دلیل خطای انسانی شایع است. از طرفی الگوهای ثابت در یک شبکه ممکن است برای شبکه دیگری از همان نوع عملی نباشد. همچنین، شبکه به طور مداوم در حال تکامل است و پویایی‌ها مانع از استفاده از یک مجموعه ثابت از الگوهایی می‌شوند که به عملکرد و مدیریت شبکه کمک کند.
\\
پیشرفت‌های کلیدی در شبکه، از جمله قابلیت برنامه نویسی شبکه از طریق شبکه نرم افزاری\footnote{\lr{SDN}}، کاربرد یادگیری ماشین را در شبکه ارتقا می‌بخشد. یکی از موانع اصلی موفقیت در شبکه‌ این است که چه داده‌هایی می تواند از آن جمع شود و چه اقدامات کنترلی را می توان در دستگاه‌های شبکه انجام داد. توانایی برنامه‌ریزی شبکه با استفاده از شبکه نرم افزاری این موانع را برطرف می‌کند. استفاده از تکنیک‌های یادگیری ماشین برای چنین مشکلات متنوع و پیچیده‌ای در شبکه مفید است. این امر باعث می‌شود یادگیری ماشین در شبکه به یک زمینه تحقیقاتی جالب توجه تبدیل شود\cite{boutaba2018comprehensive}.

\newpage

پژوهش‌های متنوعی درباره یادگیری ماشین در شبکه وجود دارد، اما این پژوهش به دلیل جامع بودن تکنیک های یادگیری ماشین تحت پوشش و جنبه های مختلف شبکه صرف نظر از فن آوری شبکه متفاوت است. در این گزارش ابتدا در مورد انواع مختلف تکنیک های مبتنی بر یادگیری ماشین، ترکیبات اساسی و تکامل آنها بررسی می‌کنیم. سپس سیر تحول و کاربردهای مباحث زیر را به ترتیب خواهیم گفت.


\begin{enumerate}
    \item مهندسی ترافیک
    \begin{itemize}
        \item پیشبینی ترافیک
        \item طبقه‌بندی ترافیک
        \item مسیریابی ترافیک
    \end{itemize}
    
    
    \item کارایی و بهینه‌سازی
    \begin{itemize}
        \item کنترل ازدحام
        \item مدیریت کیفیت سرویس\footnote{\lr{QoS}} و تجربه\footnote{\lr{QoE}}
        \item مدیریت منابع و خطا
    \end{itemize}
    \item امنیت شبکه
    
\end{enumerate}


و در نهایت فرصت‌های تحقیقاتی موجود را اشاره می‌کنیم.
